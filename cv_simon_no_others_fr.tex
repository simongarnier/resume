%%%%%%%%%%%%%%%%%%%%%%%%%%%%%%%%%%%%%%%%%
% "ModernCV" CV and Cover Letter
% LaTeX Template
% Version 1.1 (9/12/12)
%
% This template has been downloaded from:
% http://www.LaTeXTemplates.com
%
% Original author:
% Xavier Danaux (xdanaux@gmail.com)
%
% License:
% CC BY-NC-SA 3.0 (http://creativecommons.org/licenses/by-nc-sa/3.0/)
%
% Important note:
% This template requires the moderncv.cls and .sty files to be in the same
% directory as this .tex file. These files provide the resume style and themes
% used for structuring the document.
%
%%%%%%%%%%%%%%%%%%%%%%%%%%%%%%%%%%%%%%%%%

%----------------------------------------------------------------------------------------
%  PACKAGES AND OTHER DOCUMENT CONFIGURATIONS
%----------------------------------------------------------------------------------------

\documentclass[11pt,letterpaper,sans]{moderncv} % Font sizes: 10, 11, or 12; paper sizes: a4paper, letterpaper, a5paper, legalpaper, executivepaper or landscape; font families: sans or roman

\moderncvstyle{classic} % CV theme - options include: 'casual' (default), 'classic', 'oldstyle' and 'banking'
\moderncvcolor{grey} % CV color - options include: 'blue' (default), 'orange', 'green', 'red', 'purple', 'grey' and 'black'

\usepackage[utf8]{inputenc}

\usepackage{lipsum} % Used for inserting dummy 'Lorem ipsum' text into the template

\usepackage[scale=0.75]{geometry} % Reduce document margins
%\setlength{\hintscolumnwidth}{3cm} % Uncomment to change the width of the dates column
%\setlength{\makecvtitlenamewidth}{10cm} % For the 'classic' style, uncomment to adjust the width of the space allocated to your name

%----------------------------------------------------------------------------------------
%  NAME AND CONTACT INFORMATION SECTION
%----------------------------------------------------------------------------------------

\firstname{Simon} % Your first name
\familyname{Garnier} % Your last name

% All information in this block is optional, comment out any lines you don't need
%\title{}
\address{562, rue Théodore}{Montréal, Québec H1V3A9}
\mobile{514 462 6410}
%\phone{(000) 111 1112}
%\fax{(000) 111 1113}
\email{simongarnierg@gmail.com}
\homepage{github.com/simongarnier}{github.com/simongarnier} % The first argument is the url for the clickable link, the second argument is the url displayed in the template - this allows special characters to be displayed such as the tilde in this example
\homepagebis{linkedin.com/in/simongarnierg}{linkedin.com/in/simongarnierg}
%\quote{"A witty and playful quotation" - John Smith}

%----------------------------------------------------------------------------------------
\renewcommand{\listitemsymbol}{-~} % Changes the symbol used for lists
\begin{document}

\makecvtitle % Print the CV title

%----------------------------------------------------------------------------------------
%  EDUCATION SECTION
%----------------------------------------------------------------------------------------

\section{Formation}

\cventry{2012--Présent}{Baccalauréat en Informatique et génie locigiel}{\newline Université du québec à Montréal}{}{}{}
\cventry{2011}{DEC en Technique d’Intégration Multimédia}{\newline Collège de Maisonneuve}{Montréal}{}{}


%----------------------------------------------------------------------------------------
%  WORK EXPERIENCE SECTION
%----------------------------------------------------------------------------------------

\section{Expérience}

%------------------------------------------------
\cventry{2015}{Développeur d'applications mobiles}{}{Greencopper, Montréal}{}{
  \begin{itemize}
    \item[-] Développement d'applications IOS et Android natives à l'aide du framework de Greencopper
    \item[-] Écriture de script python permettant l'import de données clients
    \item[-] Développement d'un utilitaire d'analyse donnant de la valeur aux logs d'application
  \end{itemize}
}

\cventry{2014--2015}{Développeur backend et frontend}{}{Seevibes, Montréal}{}{
  \begin{itemize}
    \item[-] Ajout de fonctionnalités à une application Ruby on Rails
    \item[-] Travail sur un moteur développé en Scala
    \item[-] Manipulation de jeux de données massifs
    \item[-] Déploiement de code sur des machines EC2
  \end{itemize}
}

\cventry{2012}{Programmeur web}{}{K3 média, Montréal}{}{
  \begin{itemize}
    \item[-] Développement backend en PHP
    \item[-] Apprentissage du gestionnaire de contenu développé en open source par K3
    \item[-] Développement d’une application Facebook utilisant l’API PHP
  \end{itemize}
}

%\newpage
%----------------------------------------------------------------------
%----------------------------------------------------------------------------------------
%  COMPUTER SKILLS SECTION
%----------------------------------------------------------------------------------------
\section{Connaissances techniques}

  \cvitem{Langages}{Ruby, Python, Javascript, Scala, Java, C, Objective-C, C++, PHP, HTML + CSS}

  \cvitem{DBMSs}{PostgreSQL, MongoDB, Redis, Elasticsearch}

  \cvitem{Services}{Github, Amazon EC2}


%----------------------------------------------------------------------
%----------------------------------------------------------------------------------------
%  SKILLS SECTION
%----------------------------------------------------------------------------------------

\section{Compétences}

  \cvlistitem{Facilité à travailler en équipe et à communiquer avec des collègues.}
  \cvlistitem{Compréhension et intégration rapide de nouvelles compétences et concepts.}
  \cvlistitem{Aime relever de nouveaux défis.}


%----------------------------------------------------------------------------------------
%----------------------------------------------------------------------------------------
%  INTERESTS SECTION
%----------------------------------------------------------------------------------------

%\section{Intérêt}


%  \cvlistitem{Vélo, photographie, cinéma et cuisine.}
%----------------------------------------------------------------------------------------
%  COVER LETTER
%----------------------------------------------------------------------------------------

% To remove the cover letter, comment out this entire block

%\clearpage

%\recipient{HR Departmnet}{Corporation\\123 Pleasant Lane\\12345 City, State} % Letter recipient
%\date{\today} % Letter date
%\opening{Dear Sir or Madam,} % Opening greeting
%\closing{Sincerely yours,} % Closing phrase
%\enclosure[Attached]{curriculum vit\ae{}} % List of enclosed documents

%\makelettertitle % Print letter title

%\lipsum[1-3] % Dummy text

%\makeletterclosing % Print letter signature

%----------------------------------------------------------------------------------------

\end{document}
